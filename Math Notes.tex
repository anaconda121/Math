\documentclass[11pt,twoside]{article}

\usepackage{paperlighter}

\usepackage{amsmath}

\usepackage[utf8]{inputenc}
\usepackage{mathtools}
\usepackage[thinc]{esdiff}
\usepackage{derivative}
\usepackage{tabularx}
\usepackage{float}

\usepackage{graphicx}
\graphicspath{ {images/} }

\usepackage{hyperref}
\hypersetup{
    colorlinks=true,
    linkcolor=blue,
    filecolor=blue,      
    urlcolor=blue,
    pdfpagemode=FullScreen,
    }
\slimtitle{Phillips Exeter Academy Calculus Notes}
\slimauthor{Tanish Tyagi}

\begin{document}

\lightertitle{Phillips Exeter Academy Calculus Notes}

\lighterauthor{Tanish Tyagi$^{\dagger}$}

\lighteremail{ttyagi@exeter.edu}

\section{Math 421}

\begin{enumerate}
    \item \textbf{Important Derivatives}
    
    Page 159 - \url{https://www.exeter.edu/sites/default/files/documents/Math4_5_2021.pdf}
    
    \item \textbf{Even and Odd Functions}
    
    Odd function: $f(x) = -f(-x)$
    
    Even function: $f(x) = f(-x)$
    
    \textbf{The derivative of an odd function is an even function.}
    
    Proof:

    $f(x) = -f(-x)$ 
    
    $f'(x) = -(f'(-x) * -1)$ by chain rule
    
    $f'(x) = f'(-x)$, property of even functions
    
    \textbf{The derivative of an even function is an odd function.} This statement can be proved in a similar fashion to the above method. 

\end{enumerate}

\section{Math 431}

\begin{enumerate}

\item \textbf{Properties of Differentiable Functions}

A function is differentiable at $a$ if $f'(a)$ exists. It is differentiable on the open interval $(a, b)$ if it is differentiable at every number in the interval. If a function is differentiable at $a$ then it is also continuous at $a$. The contrapositive of this theorem states that if a function is discontinuous at $a$ then it is not differentiable at $a$.

A function is not differentiable at $a$ if its graph illustrates one of the following cases at $a$:

\begin{enumerate}
    \item \textbf{Discontinuity}
    A function is discontinuous when points are isolated from each other on a graph.
    
    \item \textbf{Vertical Tangent Line}
    The slope of a vertical tangent line is undefined, leading the function to not be differentiable at $a$ if the tangent line is vertical at $a$. 
    
    \item \textbf{Not Smooth}
    If the left and right hand limits are different at $a$, the function is therefore ``not smooth" at $a$. 
\end{enumerate}

\begin{figure}[H]
\minipage{0.32\textwidth}
  \includegraphics[scale=0.6]{discontinuity.png}
  \caption{Function that has a discontinuity}\label{fig:awesome_image1}
\endminipage\hfill
\minipage{0.32\textwidth}
  \includegraphics[scale=0.5]{not_smooth.png}
  \caption{Function that has a vertical tangent line}\label{fig:awesome_image2}
\endminipage\hfill
\minipage{0.32\textwidth}%
  \includegraphics[scale=0.5]{vertical_tangent_line.png}
  \caption{Function that is not smooth}\label{fig:awesome_image3}
\endminipage
\end{figure}

Problems: 471

\item \textbf{Antiderivatives}

\textbf{Important Antiderivative Rules:}
\begin{enumerate}
    \item Volume is the antiderivative of area for all shapes
\end{enumerate}

Problems: 473-475, 486, 

\item \textbf{Second Derivatives}

Denoted as $f''(x)$ or $\frac{\odif[order={2}]{x}}{\odif[order={2}]{y}}$. 
If $f''(a) > 0$, then the slope of the tangent line increases as x moves from less than $a$ to greater than $a$. Graph will be concave up.

If $f''(a) < 0$, the slope of the tangent line decreases as x moves from less than $a$ to greater than $a$. Graph will be concave down.

\begin{center}
\begin{tabularx}{0.8\textwidth} {
| >{\raggedright\arraybackslash}X 
| >{\centering\arraybackslash}X 
| >{\raggedleft\arraybackslash}X |}
\hline
& $f''(x) < 0$ & $f''(x) > 0$ \\
\hline
$f'(x) < 0$ & function is decreasing at a decreasing rate & function is increasing at an increasing rate\\ 
\hline
$f'(x) > 0$ & function is increasing at a decreasing rate & function is increasing at an increasing rate \\
\hline
\end{tabularx}
\end{center}

Problems: 545, 546

\item \textbf{Derivatives of Inverse Functions}

We know that $f(f^{-1}(x)) = x$. We are interested in finding $f'^{-1}x$. If we apply the chain rule, we get $f'(f^{-1}(x)) * f'^{-1}(x) = 1$. Therefore
$f'^{-1}(x) = \frac{1}{f'(f^{-1}(x))}$.

Problems: 547, 554, 

Walkthrough of 547:


\item \textbf{Critical Points}

When $f'(x) = 0$ or $f'(x) = undefined$. 

\item \textbf{Local Minimum + Maximum}

When $f'(x)$ goes from $ > 0$ to $ < 0$, the point where $f'(x) = 0$ is a local maximum. When $f'(x)$ goes from $ < 0$ to $ > 0$, the point where $f'(x) = 0$ is a local minimum. 

\item \textbf{Inflection Points}

To find inflection points, look for when the second derivative is equal to 0. Inflection point marks the change of concavity ($f''(x) > 0$ to $f''(x) < 0$ or vice versa. Once you find the point where $f''(x) = 0$, make sure to check x-coordinates before and after the point to make sure the concavity actually changes/$f''(x)$ changes signs. 

\item \textbf{Extreme Value Theorem}

If $f(x)$ is continuous for a range $a$ through $b$, then global maximum or minimum occurs at critical values ($f'(x) = 0$ or $f'(x) = undefined$ or at the endpoints $x = a$ or $x = b$.

Problems: 535, 543, 

\item \textbf{Related Rates}

Problems: 541, 561, 571

Walkthrough of 541: 

Walkthrough of 571: 

\item \textbf{Radius of Curvature} 

The radius of curvature is defined as the radius of the circular arc which best approximates the curve at a point $a$. 

General Process:
\begin{itemize}
    \item Find general form of the slope of normal line (negative reciprocal of slope of tangent line). 
    \item Use point-slope form to find equation of normal line for $P = (a, f(a))$. 
    \item Find intersection between equation of normal line at point of interest and equation of normal line from step 2. This will give you the center of curvature for the point of interest.
    \item Use distance formula to find radius of curvature by finding distance from center of curvature to point of interest. 
\end{itemize}

Problems: 507d, 527, 548, 563

\textbf{Walkthrough of 507d:}

$P = (a, cos(a))$

$f(x) = cos(x), f'(x) = -sin(x)$

Slope of Normal Line: \(\frac{1}{sin(x)}\)

Point Slope Form: y - cos(a) = \(\frac{(x-a)}{sin(x)}\)

Normal Line Equation: y = \(\frac{(x-a)}{sin(a)}\) + cos(a)

We want to find the radius of curvature at x = 0.

Therefore $a \rightarrow\text{0}$. We now need to find the intersection between lines x = 0 (normal line at x = 0) and our equation found above.

y = \(\frac{(0-a)}{sin(a)}\) + cos(a)

\(\lim_{a\to0} \frac{a}{sin(a)}\) = 1, therefore the first part of our equation equals $-1$. 

As $a \rightarrow\text{0}$, cos(a) goes to 1. 

$y = -1 + 1 = 0.$ Therefore, our center of curvature is $(0,0).$ We can now use distance formula and get a radius of curvature of 1. 

The below graph illustrates this:

\begin{figure}[H]
\centering
\includegraphics{curvature.png}
\caption{Curvature at $x = 0$ for $y = cos(x)$ }
\end{figure}


\item \textbf{Euler's Method}

Given $f'(x)$ that is not separable and hard to anti-differentiate, you can estimate any $f(x)$ value. 

We can utilize a recursive sequence to store all of our approximations. It is defined as: $y_{n+1} = y_{n} + h(f'(x))$, where $h$ is our step size. To most accurately estimate values, we can make $h$ approach 0.

Example Simulation: \url{https://replit.com/@TanishTyagi123/Eulers-Method-Math-431-550#main.py}

Problems: 550, 

\item \textbf{Solving Differential Equations}

Problems: 528, 549, 567, part of 571

Walkthrough of 567: 

Walkthrough of part of 571:


\item \textbf{Fundamental Theorem of Calculus} 

Part 1: 
$$ F'(x) = f(x) $$
$$ \int_{a}^{b} f(x) \,dx = F(a) - F(b) $$

Problems: 524, 572

\item \textbf{Riemann Sums}

Riemann sums are a way of approximating integrals. 

Let's say you want to approximate the integral $\int_{a}^{b} A(x) \,dx$: 

Let's define $\Delta x = \frac{b - a}{n}$

    \begin{enumerate}
    \item \textbf{Left Hand Riemann Sum}
$$
\lim_{n\to\infty} \sum_{k=0} ^{n-1} A(k\Delta x)\Delta x 
$$
    \item \textbf{Right Hand Riemann Sum}
$$
\lim_{n\to\infty} \sum_{k=1} ^{n} A(k\Delta x)\Delta x 
$$

    \item \textbf{Midpoint Riemann Sum} 
    \item \textbf{Trapezoidal Riemann Sums}
% $$
% \lim_{n\to\infty} \sum_{k=1} ^{n} f(kx/n) + f(
% $$
  \end{enumerate}

Problems: 558, 559, 568
  
\item \textbf{Integration}

\begin{enumerate}
    \item \textbf{Integration with Absolute Values}
    
    To get rid of the absolute value expressions, create a piecewise definition of the equation without absolute values. 
    
    Problems: 599, 623
    
    \textbf{Walkthrough of 599:}
    
    We can create a piecewise defintion of $|cos(x)|$ as such:
    \[ \begin{cases} 
    $cos(x)$ &  0 \leq $ x $ \leq \frac{\pi}{2} \\
    $-cos(x)$ &  \frac{\pi}{2} \leq $ x $ \leq \pi 
    \end{cases}
    \]
    
    \textbf{Note:} We need $-cos(x)$ for the second case because $cos(x) \leq 0$ for $\frac{\pi}{2} \leq x \leq \pi $ so $-(-) = +$
    
    Now we integrate:
    \begin{equation*}
    \int_{0}^{\pi/2} cos(x) \,dx +  \int_{\pi/2}^{\pi} cos(x) \,dx = 1 + 1 = 2
    \end{equation*}
    
    \item \textbf{Integration with Parametrics}
    
    \begin{equation*}
    \int y \,dx = \int y \frac{dx}{dt} \,dt
    \end{equation*}
    
    Problems: 611, 624, 634
    

    
\end{enumerate}

\textbf{Important Integrals:}
\begin{enumerate}
    \item \begin{equation*}
    \int_{-a}^{a} \sqrt{a^2-x^2} \,dx = \frac{\pi a^2}{2}
    \end{equation*}
    
    \item \begin{equation*}
    \int sin^2(x) \,dx = 0.5(x - sin(x)cos(x))
    \end{equation*}
    
    \item \begin{equation*}
    \int cos^2(x) \,dx = 0.5(x + sin(x)cos(x))
    \end{equation*}
\end{enumerate}
    Problems: 534, 539, 
  
  \item \textbf{Miscellaneous}
  
  Problems: 
  \begin{enumerate}
      \item 537 - Differentiating with Parametric Equations
      \item 555 - Simplifying Functions for Easy Anti-differentiation 
      \item 566 - Hard Derivative 
  \end{enumerate}
\end{enumerate}

\end{document}
