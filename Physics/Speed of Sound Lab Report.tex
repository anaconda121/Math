\documentclass[11pt,twoside]{article}

\usepackage{paperlighter}

% Recommended, but optional, packages for figures and better typesetting:
\usepackage{microtype}
\usepackage{graphicx}
\graphicspath{ {images/} }

\usepackage{subfigure}
\usepackage{booktabs} % for professional tables

\usepackage{tikz}
\usetikzlibrary{snakes}

\usepackage{wrapfig}

% Attempt to make hyperref and algorithmic work together better:
\newcommand{\theHalgorithm}{\arabic{algorithm}}

% Todonotes is useful during development; simply uncomment the next line
%    and comment out the line below the next line to turn off comments
%\usepackage[disable,textsize=tiny]{todonotes}
\usepackage[textsize=tiny]{todonotes}
\usepackage{wrapfig}

\slimtitle{Speed of Sound Lab Report}
\slimauthor{Tanish Tyagi}

\begin{document}

\lightertitle{Speed of Sound Lab Report}
\lighterauthor{Tanish Tyagi}

\section{Analysis of Standing Waves in Air}

\begin{figure}[H]
    \centering
    \includegraphics[width=0.6\columnwidth]{air_graph_with_fits.png}
    \caption{Graph of Air Column Length vs. Wavelength of Sound Wave}
\end{figure}
We can construct $L = m\lambda$ equations from the three subsets of the data, where $L$ = length of the air column and $\lambda$ = wavelength of the sound wave. This gives us the length of the air column in terms of the wavelength of the sound wave. We can manipulate the equation to get $\lambda = \frac{L}{m}$, which gives us the wavelength of the sound wave in terms of the length of the air column. This means that the data tells us the relationship between the length of the air column and the wavelength of the sound wave.
 
\textbf{Equation Fits for subsets of data:}

Black Line: $L =\frac{5}{4}\lambda$, $\lambda = \frac{4}{5}L$. 

Blue Line: $L =\frac{3}{4}\lambda$, $\lambda = \frac{4}{3}L$

Green Line: $L = \frac{1}{4}\lambda$, $\lambda = 4L$

These equations mean that one $\lambda$ is $\frac{4}{5}$, $\frac{4}{3}$, and $4$ times the length of the tube, respectively. Inside the tube, sound waves are being produced that obey the aforementioned relationships. These equations can also help us draw the resonant sound waves in the pipes. Below are the diagrams for all subsets:

% \begin{figure}[H]
% \centering

% \begin{minipage}{\textwidth}
%   \centering
%   \includegraphics[width=.33\columnwidth]{lambda_4.png}
%   \label{fig:test1}
% \end{minipage}

% \begin{minipage}{\textwidth}
%   \centering
%   \includegraphics[width=.33\columnwidth]{lambda_4_5.png}
%   \label{fig:test2}
% \end{minipage}

% \begin{minipage}{\textwidth}
%   \centering
%   \includegraphics[width=.33\columnwidth]{lambda_3_4.png}
%   \label{fig:test2}
% \end{minipage}

% \end{figure}

\begin{figure}[H]
    \centering
    \includegraphics[width=0.5\columnwidth]{lambda_4.png}
    \caption{Sound Waves Diagram for $\lambda = 4L$}
    
    \includegraphics[width=0.6\columnwidth]{lambda_4_5.png}
    \caption{Sound Waves Diagram for $\lambda = \frac{4}{5}L$}
    
    \includegraphics[width=0.6\columnwidth]{lambda_3_4.png}
    \caption{Sound Waves Diagram for $\lambda = \frac{4}{3}L$}
\end{figure}

We can see that the each of the waves have with a node at the closed end of the tube and an antinode at the open end. This is due to the boundary conditions of the wave, which are that the wave can vibrate at the open end and not at the closed end. A node is therefore formed at the closed end because the air molecules are not free to vibrate back and forth at the closed end. The opposite occurs at the open end to allow for the formation of an antinode. Since the wave has to start with a node and end with an antinode, it is not possible for any equation fits to be of the form $L = \frac{1}{2}\lambda$. An wave with this equation would have nodes at both ends, which is not possible. A standing wave diagram illustrates this below:

\begin{figure}[H]
    \centering
    \includegraphics[width=0.6\columnwidth]{L_0.5_wavelength.png}
    \caption{Standing Wave with Equation $L = 0.5\lambda$}
\end{figure}

\section{Analysis of Standing Waves in $CO_2$}

To determine the wavelength of the standing waves in $CO_2$, we can turn to our data that says the positions of the two $CO_2$ resonances are $0.12$ and $0.39$m. The positions of these resonances have to be antinodes, and they construct a standing wave that looks like this:

\begin{figure}[H]
    \centering
    \includegraphics[width=0.5\columnwidth]{0.5_wavelength.png}
    \caption{Standing Wave for $CO_2$ Data}
\end{figure}

As you can see, the distance between $0.12$ and $0.39$m is equivalent to $\frac{\lambda}{2}$. Therefore, the wavelength for the standing wave in $CO_2$ is $(0.39 - 0.12) \cdot 2 = 0.54$m. If we revisit the graph in Figure 1, we can see that the resonance positions of 0.39 and 0.12m fall into the pink and blue clusters of data points, respectively. The equation for the pink cluster is $\lambda = \frac{4}{3}L$ and $\lambda = 4L$ for the blue cluster. Using these equations, we can find the $\lambda$ for the standing wave with a resonance at $0.39$ m to be $\frac{4}{3}(0.39)$ = $0.52$m and $4(0.12)$ = $0.48$m for the standing wave with a resonance at $0.12$m. We can now use the wave equation with $f$ = $500$hz to get two wave speeds, $260$ and $240$m/s. We can average these values to get an experimental speed of $CO_2$ value of $250$m/s. 

We can use the percent error formula 
$\delta = |\frac{V_{true} - V_{observed}}{V_{true}}| \cdot 100$ to compare our experimental value with the accepted value of $260$m/s. The percent error is $\approx 3.8\%$, which is quite small for measurements that are only accurate to 2 significant figures. Additionally, it is easy to be inaccurate with recording the positions of resonances, as the water level is rising rapidly and there is a ever present risk of identifying false positives. 

We can also find the speed of sound in $CO_2$ using the established fact that the speed of sound in air is $340$m/s and the kinetic theory of gases. The kinetic energy of $N_2$ and $CO_2$ molecules should be the same by the kinetic theory of gases. The formula for kinetic energy is $\frac{1}{2}(m)(v)^2$. We need to get the mass of $N_2$ and $CO_2$ from amu to kg first. 

Mass of $N_2$: $\approx 4.6 \cdot 10^{-26}$ kg

Mass of $CO_2$: $\approx 7.3 \cdot 10^{-26}$ kg 

Therefore, the kinetic energy for the $N_2$ molecules is $KE_{N_2} = \frac{1}{2}(4.6 \cdot 10^{-26})(340)^2 \approx 2.7 \cdot 10^{-21}$ J. We can solve for the $v$ in the equation for kinetic energy in $CO_2$ to get $271.22 \approx 270$ m/s.

We can even extend our analysis to weighing molecules. For example, the speed of sound in methane, $CH_4$, is $445$m/s. Using the kinetic theory of gases, we know that the kinetic energy of molecules in $CH_4$ has to be equal to the kinetic energy of molecules in air. We know that $KE_{N_2} \approx 2.7 \cdot 10^{-21}$ J. We can set $\frac{1}{2}(m)(445)^2$ = $KE_{N_2}$, and solve for $m$ to get $m$ = $2.7 \cdot 10^{-26}$kg. We can convert this to amu and get that the molecular mass of methane is $\approx 16$ amu.

\section{Resonance and its Connection to this Lab}

Resonance is the phenomenon that causes a wave to have increased amplitude. This occurs at certain frequencies, or when the incident and reflected wave are in phase, resulting in constructive interference. In this lab, we observed the points of resonance when the length of the air column allowed for the standing wave to have a node at the closed end and antinode at the open end. It was in these scenarios that constructive inference was able to occur. The below figure shows the incident wave (in red), reflected wave (in purple), and the ``net" wave (in green).

% \begin{figure}[H]
%     \centering
%     \includegraphics[width=0.4\columnwidth]{resonance.png}
%     \caption{The Effect of Resonance}
% \end{figure}

\begin{wrapfigure}{l}{0.4\textwidth}
    \centering
    \includegraphics[width=0.4\textwidth]{resonance.png}
    \caption{The Effect of Resonance}
\end{wrapfigure}

As you can see, constructive inference occurs with then incident and reflected wave, resulting in the phenomenon of resonance taking effect and creating a wave with an amplitude twice the amplitude of incident/reflected waves. It is important to note that this only occurs when there is a node at the closed end and an antinode at the open end. If two tuning forks are held over the time at the same time and are in sync with each other, then resonance will amplify the sound wave double as much as the scenario in which there is only one tuning fork.This is because the incident and reflected waves of both tuning forks are once again in phase, resulting in resonance occurring. The figure on the next page shows this idea:

\clearpage 

\begin{figure}
    \centering
    \includegraphics[width=0.5\textwidth]{resonance_two_tuning_forks.png}
    \caption{The Effect of Resonance with Two Tuning Forks}
\end{figure} 

The amplitude produced by resonance can be generalized to $2N \cdot A_{incident}$, where $N$ is number of tuning forks, and $A_{incident}$ is the amplitude of the incident wave. If the tuning forks are not fully in sync, then only partial constructive interference will occur, and the sound will not be amplified as much by resonance as it would have been if both tuning forks were in sync.

The concept of resonance can also be seen when you blow over the top of a glass bottle and a tone is created. When you blow over the top of a bottle, you produce many different frequencies. Some of the many frequencies produced allows for the incident and reflected wave to be in phase resulting in resonance amplifying the sound wave produced by those frequencies. All the other frequencies that are generated do not change the tone created, as they are not amplified. This is similar to the concept of a velocity selector that we discussed earlier this term. Particles are fired at many different velocities, but only those fired at a certain velocity can travel through the hole. The resonant frequencies can be thought of as the particles being fired at the velocity needed to go through the hole.

\end{document}
